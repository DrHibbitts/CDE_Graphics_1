%----------------------------------------------------------------------------------------

\documentclass[paper=a4, fontsize=11pt]{scrartcl} % A4 paper and 11pt font size

\usepackage[T1]{fontenc} % Use 8-bit encoding that has 256 glyphs
%\usepackage{fourier} % Use the Adobe Utopia font for the document - comment this line to return to the LaTeX default
\usepackage[english]{babel} % English language/hyphenation
\usepackage{amsmath,amsfonts,amsthm} % Math packages
\usepackage{mathtools}
\usepackage{lipsum} % Used for inserting dummy 'Lorem ipsum' text into the template
\usepackage[margin=2.0cm]{geometry}

\usepackage{sectsty} % Allows customizing section commands
\allsectionsfont{\centering \normalfont\scshape} % Make all sections centered, the default font and small caps

\usepackage{fancyhdr} % Custom headers and footers
\pagestyle{fancyplain} % Makes all pages in the document conform to the custom headers and footers
\fancyhead{} % No page header - if you want one, create it in the same way as the footers below
\fancyfoot[L]{} % Empty left footer
\fancyfoot[C]{} % Empty center footer
\fancyfoot[R]{\thepage} % Page numbering for right footer
\renewcommand{\headrulewidth}{0pt} % Remove header underlines
\renewcommand{\footrulewidth}{0pt} % Remove footer underlines
\setlength{\headheight}{9.0pt} % Customize the height of the header

\numberwithin{equation}{section} % Number equations within sections (i.e. 1.1, 1.2, 2.1, 2.2 instead of 1, 2, 3, 4)
\numberwithin{figure}{section} % Number figures within sections (i.e. 1.1, 1.2, 2.1, 2.2 instead of 1, 2, 3, 4)
\numberwithin{table}{section} % Number tables within sections (i.e. 1.1, 1.2, 2.1, 2.2 instead of 1, 2, 3, 4)

\setlength\parindent{0pt} % Removes all indentation from paragraphs - comment this line for an assignment with lots of text

%----------------------------------------------------------------------------------------
%	TITLE SECTION
%----------------------------------------------------------------------------------------

\newcommand{\horrule}[1]{\rule{\linewidth}{#1}} % Create horizontal rule command with 1 argument of height

\title{	
\normalfont \normalsize 
\textsc{Department Of Computer Science, University of Bath} \\ [2pt] % Your university, school and/or department name(s)
\horrule{0.5pt} \\[0.2cm] % Thin top horizontal rule
\huge Inverse Kinematics Coursework \\ % The assignment title
\horrule{0.5pt} \\[0.0cm] % Thick bottom horizontal rule
}
\author{Garoe Dorta Perez, Dave Hibbitts, Ieva Kazlauskaite, Richard Shaw} % Your name

\begin{document}

\maketitle % Print the title

\section{Objective}
aaa

%------------------------------------------------

\section{Related Work}
What People do?

%------------------------------------------------

\section{Procedure}
What we did?

\subsection{Start with Matlab}
    \begin{enumerate}
    \item Issues
    \item How we solve them
    \item Why we use what we use (e.g. Inverse Jacobian)
  \end{enumerate}
\subsection{ Port the code to C++, OpenGL}
    \begin{enumerate}
    \item Issues
    \item How we solve them
  \end{enumerate}
  
\subsection{Exploration}
The following properties are to be explored.\\
1. Some way of indicating where the remote end of the linkage should move to in 3-space.\\
2. A way to change the physical properties such as rate at which the joints can change and the slow-out and slow-in of the movement of the end of the linkage as it leaves it current position and approaches the target position respectively.\\
3. Whether to model and thereby vary the flexibility of the rods.


We are allowed to work out something for one joint and explain how we'd apply it to other and so on.
%------------------------------------------------

\subsection{Summary and Conclusion}
What the final thing would look like if we had more time to work on it?

%----------------------------------------------------------------------------------------

\end{document}